%%%%%%%%%%%%%%%%%%%%%%%%%%%%%%%%%%%%%%%%%
% Simple Sectioned Essay Template
% LaTeX Template
%
% This template has been downloaded from:
% http://www.latextemplates.com
%
% Note:
% The \lipsum[#] commands throughout this template generate dummy text
% to fill the template out. These commands should all be removed when 
% writing essay content.
%
%%%%%%%%%%%%%%%%%%%%%%%%%%%%%%%%%%%%%%%%%

%----------------------------------------------------------------------------------------
%	PACKAGES AND OTHER DOCUMENT CONFIGURATIONS
%----------------------------------------------------------------------------------------

\documentclass[12pt]{article} % Default font size is 12pt, it can be changed here

\usepackage[margin=2.5cm]{geometry} % Required to change the page size to A4
\geometry{a4paper} % Set the page size to be A4 as opposed to the default US Letter

% 	Used for romanian characters
\usepackage{ucs}
\usepackage[utf8x]{inputenc}
\usepackage[english,romanian]{babel}

\usepackage{url} % Required for loading URLs
\usepackage[hidelinks]{hyperref} % Make clickable references inside doc and bookmarks
\usepackage[superscript]{cite} % Package for citations with option to make them as superscript

\makeatletter
\renewcommand\@citess[1]{\textsuperscript{[#1]}} % Re-enable square brackets before and after references
\makeatother

\usepackage{graphicx} % Required for including pictures

\usepackage{float} % Allows putting an [H] in \begin{figure} to specify the exact location of the figure
\usepackage{wrapfig} % Allows in-line images such as the example fish picture

%\usepackage{lipsum} % Used for inserting dummy 'Lorem ipsum' text into the template

\linespread{1.0} % Line spacing

\setlength\parindent{1cm} % Paragraph indentation

\graphicspath{{Pictures/}} % Specifies the directory where pictures are stored

\usepackage{pslatex} % Times New Roman font

\usepackage{listings} % Used for code snippets
\lstset{language=HTML} % Set your language (you can change the language for each code-block optionally)

\usepackage{enumitem} % For enums

\begin{document}

%----------------------------------------------------------------------------------------
%	TITLE PAGE
%----------------------------------------------------------------------------------------

\begin{titlepage}

\newcommand{\HRule}{\rule{\linewidth}{0.5mm}} % Defines a new command for the horizontal lines, change thickness here

\center % Center everything on the page

\textsc{\LARGE Universitatea Politehnica București}\\[1.5cm] % Name of your university/college
\textsc{\Large Facultatea de Automatică și Calculatoare}\\[0.5cm] % Major heading such as course name
%\textsc{\large Minor Heading}\\[0.5cm] % Minor heading such as course title

\HRule \\[0.4cm]
{ \huge \bfseries HTML5 \& CSS - Viitorul Web-ului}\\[0.4cm] % Title of your document
\HRule \\[1.5cm]

\begin{minipage}{0.4\textwidth}
\begin{flushleft} \large
\emph{Autor:}\\
Constantin \textsc{Șerban-Rădoi} % Your name
\end{flushleft}
\end{minipage}
~
\begin{minipage}{0.4\textwidth}
\begin{flushright} \large
\emph{Coordonator:} \\
Dr.Ing. Ciprian \textsc{Dobre} % Supervisor's Name
\end{flushright}
\end{minipage}\\[4cm]

{\large \today}\\[3cm] % Date, change the \today to a set date if you want to be precise

\includegraphics[height=100px]{logo-acs-text} % Include a department/university logo - this will require the graphicx package
\includegraphics[height=100px]{logo-upb}

\vfill % Fill the rest of the page with whitespace

\end{titlepage}

%----------------------------------------------------------------------------------------
%	TABLE OF CONTENTS
%----------------------------------------------------------------------------------------

\tableofcontents % Include a table of contents

\newpage % Begins the essay on a new page instead of on the same page as the table of contents 

%----------------------------------------------------------------------------------------
%	INTRODUCTION
%----------------------------------------------------------------------------------------

\section{Introducere} % Major section

Acest eseu își propune să aducă în vedere ce anume aduc nou standardele HTML5~\cite{website:html5} și
CSS3~\cite{website:css3} și ce înseamnă acestea pentru viitorul Web-ului. În prima parte se va expune o
parte din elementele ce sunt nou introduse în HTML5~\cite{website:html5} și CSS3~\cite{website:css3} și
cui sunt acestea adresate, urmând ca în a doua parte să fie cuprinsă o discuție legată de standardizare
și organizațiile implicate în acest proces.

%----------------------------------------------------------------------------------------
%	MAJOR SECTION 1
%----------------------------------------------------------------------------------------

\section{Elemente noi aduse de HTML5 și CSS3}

În ultimii ani, HTML5~\cite{website:html5} a devenit din ce în ce mai folosit în majoritatea website-urilor,
în ciuda faptului că încă nu este finalizat standardul. Același lucru poate fi spus și despre
CSS3~\cite{website:css3}, care aduce foarte multe noutăți și elemente utile pentru stilizarea paginilor,
însă nefiind standardizat, multe browsere au ajuns să implementeze în mod diferit sau, mai mult, să nu suporte
deloc anumite elemente. Aceste îmbunătățiri sunt adresate tuturor autorilor de pagini Web, dar în special
celor care cunosc deja mecanismele Web-ului. Este de dorit ca adopția HTML5~\cite{website:html5} să fie făcută
de cât mai mulți utilizatori, pentru a aduce o uniformitate mai bună prin folosirea acestor noi tehnologii.

%------------------------------------------------

\subsection{Elemente noi în HTML5} % Sub-section

În primul rând trebuie menționat ce este HTML5~\cite{website:html5}. Acesta reprezintă ultima evoluție a
standardului ce definește HTML~\cite{website:html}. Pe de o parte, el reprezintă o nouă versiune a
\textit{limbajului} HTML~\cite{website:html}, aducând o serie de noi elemente, proprietăți și comportamente,
iar pe de alta, acesta reprezintă un set mai mare de tehnologii care permit aplicații și site-uri Web.

O contribuție importantă o aduc elementele ce țin de structurarea și împărțirea unei pagini pe secțiuni,
anume \textbf{\lstinline{<section>}}, \textbf{\lstinline{<article>}}, \textbf{\lstinline{<nav>}},
\textbf{\lstinline{<header>}}, \textbf{\lstinline{<footer>}} și \textbf{\lstinline{<aside>}}.
Acestea rezolvă anumite ambiguități pe care le avea versiunea anterioară a standardului
HTML~\cite{website:html}.

Printre cele mai importante elemente aduse în HTML5~\cite{website:html5} se numără elementele ce țin de
multimedia, anume \textbf{\lstinline{<audio>}} și \textbf{\lstinline{<video>}} ce permit înglobarea de
conținut video sau audio și aduc astfel noi posibilități pentru gestionarea conținutului multimedia în
aplicațiile Web. Înainte pentru astfel de conținut era nevoie de folosirea de conținut
Flash~\cite{website:flash} sau alte tehnologii similare, care fac parte din standard.

Au fost aduse îmbunătățiri și asupra formularelor introduse cu ajutorul tag-ului \textbf{\lstinline{<form>}}.
Mai precis s-au adăugat noi valori pentru atributul \textbf{\lstinline{type}} din cadrul tag-ului
\textbf{\lstinline{<input>}}. Aceste valori sunt: \textbf{\textit{search}}, \textbf{\textit{email}},
\textbf{\textit{tel}} și \textbf{\textit{url}}. Prin intermediul acestor tipuri se pot introduce mai ușor
câmpuri de adresă de email, telefon, url sau câmpuri de căutare.

În afară de elementele și atributele mai sus menționate au mai fost adăugate multe alte noi elemente semantice
precum \textbf{\lstinline{<mark>}}, \textbf{\lstinline{<figure>}}, \textbf{\lstinline{<figcaption>}},
\textbf{\lstinline{<data>}}, \textbf{\lstinline{<time>}}, \textbf{\lstinline{<output>}},
\textbf{\lstinline{<progress>}}, sau \textbf{\lstinline{<meter>}} și \textbf{\lstinline{<main>}}, sporind
numărul elementelor HTML5~\cite{website:html5} valide.

În ceea ce privește tehnologiile de comunicare, WebSockets~\cite{website:websockets} au avut o contribuție
extrem de importantă permițând mult mai ușor menținerea unor sesiuni de comunicație dintre browser și un
server. Tot aici ar fi de menționat și WebRTC~\cite{website:webrtc}, care este un standard de comunicație în
timp real.

O altă îmbunătățire majoră a fost adusă la partea de grafică 2D și 3D prin introducerea tag-ului
\textbf{\lstinline{<canvas>}} și a bibliotecii grafice WebGL~\cite{website:webgl} ce permit randarea de
elemente grafice 2D și 3D direct într-un browser compatibil, fără a fi nevoie de alte tehnologii intermediare.

%------------------------------------------------

\subsection{Elemente noi în CSS3} % Sub-section

CSS~\cite{website:css} a fost extins pentru a putea stiliza elementele într-un mod mult mai complex decât se
putea realiza până acum. Acest lucru este de regulă referit ca CSS3~\cite{website:css3}, deși
CSS~\cite{website:css} nu mai este reprezentat de o specificație monolitică, iar diferite module nu se află
la același nivel 3, spre exemplu unele se află la primul, în vreme ce altele se afla la al patrulea nivel,
toate celelalte nivele fiind folosite.

Au fost adăugate noi metode de stilizare a fundalului, de exemplu prin utilizarea umbrelor în cadre, folosind
\textbf{\textit{\lstinline{box-shadow}}} și de asemenea pot fi setate mai multe fundaluri.

De asemenea, au fost aduse contribuții marginilor, acum putând folosi imagini pentru a stiliza marginile,
folosind \textbf{\textit{\lstinline{border-image}}} și proprietățile asociate acestuia, iar marginile
rotunjite sunt suportate prin intermediul proprietății \textbf{\textit{\lstinline{border-radius}}}.

De departe cea mai importantă contribuție adusă în CSS3~\cite{website:css3} au fost Tranzițiile
~\cite{website:transitions} și Animațiile CSS~\cite{website:animations} prin intermediul cărora se pot crea
mult mai ușor efecte complexe ce pot îmbogăți experiența utilizatorilor pe site-urile unde sunt folosite.

Pe lângă aceste elemente au fost aduse îmbunătățiri și în domeniul tipografic, autorii site-urilor putând
controla împărțirea cuvintelor în silabe și a modului în care textul este așezat pe rânduri. Mai mult,
acum se pot adăuga umbre și se pot controla mai ușor decorațiunile asupra textului.

%----------------------------------------------------------------------------------------
%	MAJOR SECTION 2
%----------------------------------------------------------------------------------------

\section{Standardizare. Cine se ocupă?} % Major section

Pentru a se încerca să existe o sintaxă uniformă pentru toate browserele există un efort de standardizare
a acestor tehnologii, de care se ocupă HTML Working Group~\cite{website:htmlwg}, un grup de persoane cu
expertiză în domeniul web. Lista de membri din acest grup~\cite{website:htmlwgmembers} cuprinde experți
din cadrul mai multor organizații, cea mai importantă fiind W3C~\cite{website:w3c}. Alte organizații cu un
număr semnificativ de membri, în ordine alfabetică, sunt:

\begin{itemize}[nosep] % No vertical spacing
	\item Adobe Systems, Inc.~\cite{website:adobeinc}
    \item Apple, Inc.~\cite{website:appleinc}
	\item Baidu, Inc.~\cite{website:baiduinc}
	\item Google, Inc.~\cite{website:googleinc}
	\item Microsoft Corporation~\cite{website:microsoft}
	\item Mozilla Foundation~\cite{website:mozilla}
\end{itemize}

În Decembrie 2012 a fost realizată specificația Candidate Recommendation pentru standardul
HTML5~\cite{website:html5cr} de către W3C, urmând ca ulterior să se stabilizeze. Planul pentru anul 2014
pentru standardizarea HTML5~\cite{website:html5-2014-plan} își propune ca la sfârșitul Q4 să fie
publicată versiunea finală a standardului. Mai mult, tot la sfârșitul Q4 se dorește realizarea unei
specificații Candidate Recommendation pentru versiunea ulterioară, HTML5.1~\cite{website:html5.1-milestones},
care se dorește a îngloba tot ceea ce oferă deja standardul HTML5~\cite{website:html5}, alături de alte
elemente care nu erau încă suficient de stabile pentru a fi incluse în standard și eventuale amendamente
și rezolvări ale bug-urilor existente.

Suportul pentru HTML5~\cite{website:html5} în browserele existente~\cite{website:html5-support} a început
să ajungă la un nivel destul de bun, Chrome și Firefox situându-se cel mai bine la numărul de feature-uri
implementate. Internet Explorer (versiunea 11), deși este fundaș, suportă deja un număr semnificativ de
elemente. Rămâne de văzut care va fi situația la sfârșitul acestui an, după ce se stabilește standardul în
forma sa finală.

%----------------------------------------------------------------------------------------
%	CONCLUSION
%----------------------------------------------------------------------------------------

\section{Concluzie} % Major section

Web-ul pare să se îndrepte într-o direcție interesantă prin intermediul introducerii noilor tehnologii.
Este un lucru bun că se încearcă standardizarea acestora, deoarece momentan există diverse inconsistențe
între implementările din diverse motoare de browsere (Webkit versus Geko versus Trident). Existența acestor
inconsistențe îngreunează munca dezvoltatorilor Web, deoarece aceștia trebuie să considere diferite
implementări specifice pentru fiecare motor în parte. În momentul definitivării standardului, acest lucru
se va diminua treptat, pe măsură ce vor fi implementate toate specificațiile conform cu standardul în
toate browserele. O altă problemă comună este cea de backwards compatibility, în special pentru browsere
precum cele din gama Internet Explorer 8 sau mai vechi, deoarece acestea nu suportau nici măcar vechiul
standard așa cum ar fi trebuit, și acest lucru a dus la provocarea coșmarurilor pentru dezvoltatori.

În ceea ce privește utilitatea adusă de noile tehnologii, aș putea spune că este una extrem de importantă,
creând noi modalități de a arăta informația și de a interacționa cu ea, și, mai mult, se pare că s-a creat
un trend prin care se face o trecere dinspre aplicații desktop către domeniul Web, pentru toate platformele.
Mai mult, trendul mobilității contribuie la acest lucru prin dorința de a avea o experiență adecvată pe
cât mai multe dispozitive în momentul folosirii aplicațiilor web atât pe dispozitive mobile, cu ecran redus,
cât și pe desktop-uri sau tablete, la care există un ecran mare.

%----------------------------------------------------------------------------------------
%	BIBLIOGRAPHY
%----------------------------------------------------------------------------------------
\newpage % Begin Bibliography on a new page

\bibliography{bibliography}{}
\bibliographystyle{plain}

%----------------------------------------------------------------------------------------

\end{document}